%%%%%%%%%%%%%%%%%%%%%%%%%%%%%%%%%%%%%%%%%%%%%%%%%%
%% Bibliographical Information 
%%%%%%%%%%%%%%%%%%%%%%%%%%%%%%%%%%%%%%%%%%%%%%%%%%
\begin{flushleft}{\bf Bibliographische Daten} \\\rule{120mm}{0.5mm}\end{flushleft}
\mytitle \\
(\mytitleDE)\\
\melast, \mefirst\\
Universit\"at Leipzig, Dissertation, \year \\
% NOTE: the number below are placed into files, which is useful, if
% you decide to generate them automatically (as detailed below). If
% you are the manual type, you can also just count them and insert a number.
%\input{\mypaths pages.num}Seiten, \input{\mypaths figs.num}Abbildungen, \input{\mypaths refs.num}Referenzen\\

% Automatizing the counters:
% General: since you need the counters here, but the information is
% only present in your thesis file, one needs to write them to disk
% (the *.num files above)  during typesetting the thesis. 
% You can do so, by using the newfile latex package (just google it)
% and insert the following lines in you main file:
%
% \newoutputstream{pages}
% \openoutputfile{pages.num}{pages} 
% \addtostream{pages}{\arabic{page}}
% \closeoutputstream{pages}
% 
% They need to be adapted for each of the following counters:
%
% Page counter:
% The page counter is already defined in latex: as you see above, its
% value is written do disk by issueing \arabic{page} in the newfile
% constructs.
%
% Figure counter: 
% If you have defined a figure environment, e.g. 
% \newcommand{\Figure}[3][htp]{
%  \addtocounter{AllFigures}{1}
%  \begin{figure}[#1]
%    \centering % if Figure Labels don't work remove this
%    \includegraphics[width=\textwidth]{Figures/FIG_#2.pdf}
%    \caption{#3}  
%    \label{#2}
%  \end{figure}}
%
% If you have not defined a figure environment, use query replace, e.g.
% end{figure} by addtocounter{AllFigures}{1} \\end{figure}
% If your thesis is spread out over multiple .tex files use reftex-query-replace-document
%
% then (as above), you can just insert the counter increment.
% Before this definition you need to define the counter:
% \newcounter{AllFigures}
% Then you need to adapt the writefile lines above to use the counter
% AllFigures and write to the file figs.num
%
% Ref counter:
% To automatically count the number of references, it seemed easiest
% to insert a counter into the .bbl file produced by BiBTeX. Since the
% .bbl-file is regenerated often, you need to insert the counter in
% the definition of your BiBTeX-style, i.e. the corresponding .bst
% file. Which one you use is shown in the BiBTeX output. Once you have
% found it, look for the place where the \bibitem is inserted, e.g.
% 
% FUNCTION {output.bibitem}
%{ newline$
%  "\addtocounter{Cites}{1} \bibitem[" write$
%  label write$
%  "]{" write$
%  cite$ write$
%  "}" write$
%  newline$
%  ""
%  before.all 'output.state :=
%}
%
% where the \addtocounter{Cites}{1} was newly inserted. And again the
% counter Cites needs to be defined somewhere in your TeX code:
% \newcounter{Cites}
%
% OK, once you have confirmed that these number are written to the
% hard drive and that the relative paths are correct, you should never
% have to worry about counting references, figures, or pages again. 


%%% Local Variables: 
%%% mode: latex
%%% TeX-master: "dissdocuments.tex"
%%% End: 