\begin{center}
{\Large {\bf Ver\"offentlichungen und Vortr\"age}} \vspace{0.5cm}
\end{center}

This thesis presents our contributions to RNA secondary structures' computational methods for the folding and inverse folding problems. They were obtained in collaboration with my advisor Matteo Smerlak, Vaitea Opuu and Vincent Messow. Most of the ideas and figures have appeared previously in the following publications: 

\begin{itemize}

\item \textbf{Nono SC Merleau} and Matteo Smerlak (2021). \emph{A simple evolutionary algorithm guided by local mutations for an efficient RNA design}. In: \textit{Proceedings of the Genetic and Evolutionary Computation Conference.} pp. 1027-1034. %(\textcolor{green}{Published})

\item Vaitea Opuu, \textbf{Nono SC Merleau}, Vincent Messow, and Matteo Smerlak(2021). \emph{RAFFT: Efficient prediction of RNA folding pathways using the fast Fourier transform}. In: \textit{PLoS Comput. Biol.}

\item \textbf{Nono SC Merleau}  and Matteo Smerlak (2022). \emph{An evolutionary algorithm for inverse RNA folding inspired by Lévy flights}. In: \textit{BMC Bioinformatics}, 23.1

\end{itemize}
(* markiert gleichen Beitrag der Autoren)\\[.25cm]
%Weitere drei Manuskripte befinden sich derzeit in Begutachtung oder in Vorbereitung.\\[.25cm]


%{\large Vortr\"age}

In addition to these works in RNA folding and inverse folding, I studied the fragility of RNA viruses during my PhD using multi-agent evolutionary algorithm simulations. I also contributed to various works in natural language processing and multi-agent simulations for Holonification models. 
None of these investigations,

\begin{itemize}
	
	\item Igor Haman Tchappi, Stéphane Galland, Vivient Corneille Kamla, Jean-Claude Kamgang, \textbf{Cyrille Merleau S Nono}, and Hui Zhao (2019). \emph{Holonification model for a multilevel agent-based system}. In: \textit{Personal and Ubiquitous Computing} 23(5).
	
	\item Ivan P Yamshchikov, \textbf{Cyrille Merleau Nono Saha}, Igor Samenko, Jürgen Jost (2020). \emph{It Means More if It Sounds Good: Yet Another Hypothesis Concerning the Evolution of Polysemous Words}. In: \textit{Proceedings of the 5th International Conference on Complexity, Future Information Systems and Risk (COMPLEXIS 2020)}, pages 143-148.
	
	\item  \textbf{Nono SC Merleau}, Sophie Pénisson, Philip J Gerrish, Santiago F Elena, and Matteo Smerlak (2021). \emph{Why are viral genomes so fragile? The bottleneck hypothesis}. In: \textit{PLoS. Comput Biol}. 17(7). %(\textcolor{green}{Published})
	
	
\end{itemize}
will be addressed in this manuscript.
%\cite{Englitz:2008p1,Tolnai:2008p1980,Tolnai:2008p713,Haustein:2008p1533}

%\bibliography{Literature/Papers}
%\bibliographystyle{apalike}

%%% Local Variables: 
%%% mode: latex
%%% TeX-master: "DissDocuments"
%%% End: 