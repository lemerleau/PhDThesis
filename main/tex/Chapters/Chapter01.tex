%************************************************
\chapter{RNA world}\label{ch:introduction}
%************************************************
A short chapter intro here.

\section{Macromolecular polymers: From DNA to RNA}
A short story on how the attention has moved from DNA to RNA. 

This section aim at: 

\begin{itemize}
    \marginpar{You can use these margins for summaries of the text
    body\dots}
    \item Defining DNA and stating its biological role
   	
   	\item History from DNA to RNAs

    \item How the covid19 has even amplified the attention on RNAs
\end{itemize}
In total, this should get you started in no time.

\section{Biochemistry of RNA molecules}\label{sec:rna_biochemical}
Here we will provide a biochemical definition of RNA molecules. 

\begin{itemize}
    \item Define each nucleobase composing RNA. 
    
    \item Define the base pairing biochemistry 
    
    \item Introduce pseudoknot pairing biochemical concepts.

\end{itemize}


\section{non-coding RNAs and their biological implications}\label{sec:custom}

This section may be included in the section 1.1.  Nevertheless, this will provide a particular introduction to some non-coding RNAs and underlay their biological significances. e.g. Aptamers \& Riboswitches, SELEX, etc...

\section{Formal definitions}

the section aim at providing formal definitions on RNA sequences, structures, network, etc...used in this thesis. 

\textbf{Definition 1}( RNA sequence) :  More formally, an RNA molecule consists of an ordered sequence of nucleotides that can be represented as:~ \[x\ =\ \left(x_1,...,x_L\right)\ where\ x_i\in\left\{A,C,G,U\right\}\]~
\(x_{ }\)~is often known as the primary structure of RNA.

\textbf{Definition 2} (RNA pseudoknot-free secondary structure): For simplicity, a secondary structure of ~such sequence~\(x_{ }\)~is a list of base pairs~\(\left(i,j\right)\)~on~\(x\) with the following constraints:~

\begin{enumerate}
	\item
	A nucleotide (sequence position) can only belong to a single pair,
	\item
	No pseudoknots: No pairs~\(\left(i,j\right)\)~and
	\(\left(k,l\right)\)~with~\(i<k<j<l\),
	\item
	~If \(\left(i,j\right)\)~is a pair then~\(x_ix_j \in \left\{GC,CG,AU,UA,GU,UG\right\}\),
	\item
	~If \((i,j)\)~is a base pair, then \(j-i>3\).
\end{enumerate}

\textbf{Definition 3} (MFE secondary structure):

\textbf{Definition 4} ( Structure Ensemble):

\textbf{Definition 5} (Secondary structure probability)

\textbf{Definition } (Secondary structure loop) : 

\textbf{Definition 6} (Base pair probability): 

\textbf{Definition 7} (Partition function of RNA) : 

\textbf{Definition 8} (Base pair probability matrix): 

\textbf{Definition 9} (Neutral set of RNA sequences) : 

\textbf{Definition 10} (Neutral Network): 

\textbf{Definition 11}  (PPV) : 

\textbf{Definition 12} (FPV) : 

\textbf{Definition 13 } (FFT) : 

\textbf{Definition 14} (Hamming Distance between two SS): 

\textbf{Definition 15} Ensemble defect (ED) \cite{zadeh2011nucleic}: Here, we use the ED as a second objective function for refinement after having at least one sequence that folds into the target in the current population. It is defined as follows: ~

\begin{equation}
\label{ed}
\begin{split}
ED(\phi, \sigma*) &= \sum_{\sigma \in \Gamma}{p(\phi, \sigma)d(\sigma, \sigma*)}\\
&= L - \sum_{1<i,j<L} P_{i,j}(\phi)S_{i,j}(\sigma*)
\end{split}
\end{equation}

where~\(P_{i,j}\)~is the base pair probability matrix and~\(S(s)\)~is the structure matrix with entries~\(S_{i,j} \in  \{ 0, 1\}\). If the structure~\(s\)contains pair~\(\{i ,j\}\), then~\(S_{i,j}(s) = 1\)~otherwise \(S_{i,j}(s) = 0\).

\textbf{Definition 16} Normalized Energy Distance (NED): the difference between
the energy of a given sequence~\(\phi\)~evaluated to fold
into a target structure~\(\sigma*\)~and the minimum free energy
of the sequence in its structural ensemble~\(\Gamma\).~ The value is normalized over all the sequences in a given population $P$.  


\begin{equation}
\label{ned}
NED(\phi, \sigma*) = [1-\Delta E_{norm}(\phi, \sigma*)]^p \text{   } \forall p>1
\end{equation}
where,
\begin{equation}
\Delta E_{norm}(\phi, \sigma*) = \frac{\Delta E(\phi, \sigma*) }{\sum_{\phi \in P}{\Delta E(\phi, \sigma*)}}
\end{equation}
and,
\begin{equation}
\Delta E(\phi, \sigma*) = E(\phi, \sigma*) - \arg \min_{s \in \Gamma} E(\phi, s)
\end{equation}

\textbf{Definition 17} (Fitness landscape) : 


\textbf{Definition 18} (Local minima): 

\textbf{Definition 19} (Global minima): 

\textbf{Definition 20} (Lévy Flights): 

\textbf{Definition 21} (Local search): 

