%*****************************************
\chapter{Introduction to RNA folding}\label{ch:folding}
%The function of non-coding \acp{RNA} is largely determined by their three-dimensional structure \cite{cech2014noncoding}. For instance, we can analyze the catalytic function of ribozymes in terms of basic structural motifs, e.g. hammerhead or hairpin structures \cite{doherty2001ribozyme}. Other \acp{RNA}, like riboswitches, involve changes between alternative structures  \cite{vitreschak04_ribos}. Therefore, understanding the relation sequence and structure is a central challenge in molecular biology. In the last $20$ years, many different methods for determining the \ac{RNA} structures of molecules have emerged: from experimental lab methods to computational approaches. For experimental lab methods, X-ray crystallography and the nuclear magnetic resonance (NMR)  are the most accurate approaches to offer structural information at a single base-pair resolution. Both experimental methods are often characterized by high experimental cost and low throughput. In addition to those limitations, \ac{RNA} molecules are volatile and difficult to crystallize. Despite the development of more sophisticated techniques to infer the state of nucleotides in \ac{RNA} molecules using enzymatic \cite{kertesz2010genome, underwood2010fragseq} or chemical probes \cite{tijerina2007dms, wilkinson2006selective} coupled with next-generation sequencing \cite{bevilacqua2016genome, tian2016rna}, most of them can only capture \ac{RNA} structures \textit{in vitro} which mostly differ from the \textit{in vivo} structure conformations. Experimentally, only a tiny fraction of known \acp{ncRNA} has been determined \cite{rnacentral2017rnacentral}. Because measuring the structure of \acp{RNA} experimentally is very difficult and expensive, computational approaches play a central role in the analysis of natural \acp{RNA} \cite{seetin2012rna, fallmann2017recent}, and are an important alternative to experimental approaches. In nature, \acp{RNA} fold into secondary structures before folding into higher-level (tertiary and quaternary) structures \cite{brion1997hierarchy,tinoco1999rna}. This separation of time scales justifies focusing on the prediction of secondary structure; evidence suggests that the resulting high-level structures are indeed largely determined by the \ac{RNA}'s secondary structures. This chapter provides an overview of computational folding methods of \ac{RNA} molecules into a secondary structure. Two techniques will be reviewed: statistical approaches such as machine learning and score-based methods.
We provided some motivations for studying \acp{ncRNA} and introduced their bioinformatic concepts in the introduction. We also highlighted the relationship between the structure of \acp{ncRNA} and their functions. The functions of \acp{ncRNA} and their lengths usually distinguish them, and several \ac{ncRNA} classes were presented. Identifying the \ac{ncRNA} functions is challenging, though there is a widespread expectation that their functions are largely determined by their structures. The process of determining the \ac{RNA} structure is often termed \ac{RNA} folding. Experimental methods that determine the secondary structure of such molecules are usually expensive. Many computational methods have been developed in the last decades as alternatives. This chapter overviews computational methods for predicting \ac{RNA} secondary structures. Two techniques will be reviewed: statistical approaches such as machine learning and score-based methods.


\section{Stability and prediction of RNA secondary structures}
%\graffito{Note: The content of this chapter is just some dummy text.
%It is not a real language.}

The mapping from \ac{RNA} sequences to their corresponding secondary structure defines the folding of \ac{RNA} molecules. \ac{RNA} folding is, therefore, a process by which a linear \ac{RNA} sequence acquires a secondary structure through intra-molecular interactions. The nature of those interactions defines the thermodynamic stability of the secondary structure. Throughout this dissertation, we will denote the thermodynamic stability of a structure $\sigma$ by $\Delta G_{\sigma}$, which is the free energy difference with respect to the completely unfolded state.  This section provides an intuition on how the \textit{free energy} of an RNA secondary structure is computed based on the definitions and concepts introduced in \autoref{ch:introduction}. Furthermore, it introduces the problem of RNA secondary structure prediction and an overview of existing techniques. 


In predicting biologically relevant structures, most computational methods search for structures that minimize the free energy function $\Delta G$ (i.e. the \ac{MFE} structure). Therefore, the prerequisite to efficiently computing the \ac{MFE} secondary structure is the computation of the free energy for any given secondary structure $\mathcal {S}$. The calculation of the \ac{RNA} structure free energies starts by decomposing each structure into components called loops (See Definition \autoref{def:loops}). The loop decomposition allows building the basis of the standard energy model for \ac{RNA} secondary structures called the \ac{NN} model \cite{turner09_nndb}. The total free energy of a secondary structure is assumed to be a sum over its constituent loops according to the additivity principle \cite{dill97_addit_princ_bioch} (see Definition \ref{def:free_energy}). Therefore, this structure decomposition allows an efficient \ac{DP} algorithm to determine the \ac{MFE} pseudoknot-free structure of a sequence $\phi$ in the structure space $\Sigma_{\phi}$. 

The \ac{DP} is a computer programming method developed by Richard Bellman in the early 1950s \cite{bellman1957dynamic}, and it has found applications in various fields, including the \ac{RNA} secondary structure prediction. It consists of simplifying a complicated problem by breaking it down into simpler sub-problems in a recursive manner. When sub-problems can be nested recursively inside larger problems so that \ac{DP} methods are applicable, then there is a relation between the value of the larger problem instance and the values of the sub-problems. 

For example, let us consider the definition of secondary structure $\mathcal{S}$ introduced in the previous chapter (Definition \autoref{def:rna_structure}) and its string representation $\sigma$. When considering a substructure $\sigma[i:j]$ within the sequence interval $\phi[i:j]$, there are only two alternatives to how position $i$ may contribute to $\sigma[i:j]$. Either $i$ does not pair with any other position, or it pairs with another nucleotide $k$ with $i<k \leq j$. In the first situation, $\sigma[i : j]$ consists of the base-pairs in the subsequence $\sigma[i+1: j]$ only. The formation of a base-pair $(i, k)$, however, subdivides the structure into two parts, one enclosed by $(i, k)$, namely $\sigma[i + 1 : k - 1]$, and the other one, $\sigma[k + 1 : j]$. Thus, $\mathcal{S} = proc\{\sigma[i + 1 : k - 1] \cup  \sigma[k + 1 : j]\} \cup \{(i, k)\}$, where the $proc$ is the recursive procedure. Since condition (3) of definition \autoref{def:rna_structure} ensures that the position $(i,j)$ can not contain base-pairs that cross $(i, k)$ (or at least in the pseudoknot-free situation), the two shorter substructures $\sigma[i + 1 : k - 1]$ and $\sigma[k + 1: j]$ can be treated independently for a large variety of purposes.

This observation has led to a recursive decomposition scheme for \ac{RNA} secondary structures, which is the basis of the large variety of \ac{DP} approaches that solve \ac{RNA} secondary structure prediction problems. The first \ac{DP} algorithm was then proposed by Nussinov and Jacobson \cite{nussinov1980fast} to find the structure with the maximum base-pairs. A few years after, Zucker and Stieger \cite{zuker1981optimal} extended Nussinov's algorithm to a more realistic scoring model based on free energy, the \ac{NN} model. Almost all score-based methods rely on the same \ac{DP} algorithm, but the decomposition scheme and the scoring model could differ from one to another. When predicting structures with non-canonical base-pairs, some other scoring schemes are used, such as nucleotide cyclic motifs score system \cite{zu2011folding,parisien2008mc,dallaire2016exploring} or equilibrium partition function \cite{sloma2017base}. 

In addition to score-based methods, we have comparative sequence analysis methods, which are the most computationally accurate for determining \ac{RNA} secondary structures \cite{gutell2002accuracy, madison1966nucleotide}. Using the set of homologous structures, the comparative method allows finding base-pairs that covary to maintain \ac{WC} and wobble bases of a given sequence $\phi$ \cite{gutell1985comparative}. The first comparative method predicting a common secondary structure conserved in the given homologous sequence set was developed by Han and Kim in the early nineteenth century, and it was based on comparative phylogenetic analysis. 

When neglecting the special base-pairs (or pseudoknots) and the weak interactions, the running time of both approaches (sore-based and comparative analysis) is usually $O(L^3)$ (Where $L$ is the \ac{RNA} sequence length) and thus prohibitingly slow for longer sequences.. Many other comparative analysis methods and variations of score-based methods were also proposed to improve computational time. More recently, a heuristic method such as \texttt{LinearFold} allows achieving good \ac{RNA} folding performance in a linear time ($O(L)$). 

When pseudoknots are considered, the loop decomposition of a secondary structure and the energy rules break down. Although we can assign reasonable free energies to the helices in a pseudoknot and even to possible coaxial stacking between them, it is impossible to estimate the effects of the new kinds of loops created. Base triples pose an even greater challenge because the exact nature of the triple cannot be predicted in advance, and even if it could, we have no data for assigning free energies.
Nevertheless, there are existing techniques that approximate the energies of pseudoknot loops and allow the dynamic programming technique to tackle the \ac{RNA} folding with pseudoknots. However, the time complexity still remains the main problem. Using the \ac{DP} technique for the pseudoknot structure prediction, the time complexity goes up to $O(L^6)$ for the exact prediction. But for heuristic methods such as \texttt{IPKnot} \cite{sato2011ipknot} and \texttt{Hotknots} \cite{ren2005hotknots}, the running time can be reduced down to $O(L^4)$. 

Despite the advanced development of computational tools for \ac{RNA} folding, it's challenging to understand the folding mechanism fully. In contrast to score-based and comparative analysis methods, machine learning methods are data-driven methods that require no knowledge of the folding mechanism. Nevertheless, the requirement of \ac{ML}-based methods is a large amount of training data on which they can learn. In the last few decades, \ac{ML} methods have been used for many aspects of \ac{RNA} secondary structure prediction methods to improve the prediction performance and overcome the limitations of existing methods. However, they did not replace the mainstream score-based methods with respect to accuracy and generalization. In addition to some overfitting concerns, \ac{ML}-based methods cannot give dynamic information on the \ac{RNA} folding process since little data are available on structural dynamics. In addition, the training data used in \ac{ML}-based methods are mostly obtained through phylogenetic analyses. Consequently, their prediction may be biased due to the \textit{in vivo} third elements. The following subsections provide a detailed description of some of the recent \ac{ML}-based and score-based tools for secondary structure prediction.

In sum, computational methods usually consider the \ac{MFE} secondary structure as the most biologically relevant one. Predicting the \ac{MFE} structure consists of solving a free energy optimization problem in the case the scoring function is the free energy. Existing methods for RNA secondary structures prediction can be clustered into three main categories: the scored-based, comparative sequence analysis and \ac{ML} methods. The \ac{DP} technique is one of the most widely used score-based methods, but they are usually less accurate than the comparative methods. In contrast, \ac{ML} methods are more recent and still under intensive improvements. The following section will overview some existing tools and highlight their limitations. 

\subsection{ MFE prediction tools for pseudoknot-free \ac{RNA} sequences using a score-base method}
The score-based methods often assume that the native or biological \ac{RNA} structure is the one that minimizes/maximizes the overall total score, depending on the hypotheses made on the \ac{RNA} folding mechanism. In the pseudoknot-free \ac{MFE} prediction, where the special and weak interactions are neglected, the folding problem is less complex, and the scoring model is simply the free energy. Hence, the issue of \ac{RNA} secondary structure prediction becomes an optimization problem that aims at finding the best-scoring structure $\mathcal{S}^{MFE}$ by minimizing a scoring function $\Delta G$. We write

\begin{equation}
\mathcal{S}^{MFE} =  \textbf{argmin}_{\mathcal{S} \in \Sigma_{\phi}} \Delta G(\mathcal{S}, \phi) 
\end{equation}

where $\Sigma_{\phi}$ is the set of all possible pseudo-knot free secondary structures for the sequence $\phi$ of length $L$ and, $\Delta G(\mathcal{S}, \phi)$ the free energy of the structure $\mathcal{S}$ evaluated for the sequence $\phi$.

Since each possible structure can be uniquely and recursively decomposed into smaller components (or loops) with independent free energy contributions, the \ac{DP} is best suited for most of the following tools presented here.
%Equation \textcolor{red}{4(to fill up)}  shows the scoring function used in the \ac{DP} tools presented in this work.

\begin{itemize}
	\item \texttt{Unfold} \cite{zuker1981optimal, zuker1984rna}: It is the successor of the original  \texttt{mfold} program which was the first realistic implementation of the \ac{DP} for secondary structure predictions with a score based on the loop energy parameters and a worse case time complexity of $O(L^3)$. The initial version was an improvement of the simplest \ac{DP} for secondary structure prediction known as the \textit{maximum circular matching problem}. The authors demonstrated that the loop-based energy model is also amenable to the same algorithmic ideas. With McCaskill's algorithm \cite{mccaskill1990equilibrium}, for computing the partition function of the equilibrium ensemble of \ac{RNA} molecules, more efficient implementations of the initial program with accurate thermodynamic modelling have been provided. The latest implementation is known as \texttt{Unfold}.
	
	\item \texttt{RNAStructure} \cite{matthews1998updated,reuter2010rnastructure}: The software first appeared in 1998 as a reimplementation of the program \texttt{mfold} with improved thermodynamic parameters. In its initial version, four major changes were made in \texttt{mfold}: (1) an improvement on the methods for forcing base-pairs; (2) a filter that removed isolated \ac{WC} or wobble base-pairs has been added; (3) the energy parameter for interior, internal and hairpin loops were incorporated; (4) a new model for coaxial stacking of helices. It predicts the lowest free energy structure and a set of low energy structures. The new implementation also provided a user-friendly graphical interface for Windows operating system. Subsequently, the first implementation was extended to include biomolecular folding; an algorithm that finds low free energy structures common to two sequences; the partition function algorithm and all free energy structures, and the constraints with enzymatic data and chemical mapping data. The recent version includes the partition function computation for secondary structures common to two sequences and can perform stochastic sampling of common structures \cite{harmanci2009stochastic}. Additionally, it contains \texttt{MaxExpect}, which finds maximum expected accuracy structures \cite{lu2009improved}, and a method for removal of pseudoknots, leaving behind the lowest free energy pseudoknot-free structure. 
	
	\item \texttt{RNAfold} \cite{hofacker1994fast,lorenz11_vienn_packag}: It is one of the most used and efficient folding tools. It computes the \ac{MFE} secondary structure using an efficient \ac{DP} scheme and backtraces an optimum structure. It also allows computing the partition function using McCaskill's algorithm, the matrix of base-pairing probabilities, and the centroid structure. It is part of the \texttt{ViennaRNA Package}. Since its first version, it aims at suggesting an efficient implementation of Zucker's algorithm with more flexibility on the folding constraints. Many other versions have been released, including a \ac{GPU} implementation. The latest stable release of the \texttt{ViennaRNA Package} is Version 2.5.0.
	
	%\item \texttt{RNAshape}\cite{steffen2006rnashapes}:
	\item \texttt{LinearFold} \cite{huang2019linearfold}: For many decades, the \ac{DP} techniques have been the most accurate and fast at predicting pseudoknot-free structure for short input \ac{RNA} sequences. But for long sequences, the prediction remains challenging because of the computational time and the lack of accurate thermodynamic energy parameters. In contrast to traditional \ac{DP} methods which are often bottom-up, \texttt{LinearFold} is a left-to-right \ac{DP}. The left-to-right \ac{DP} consists of scanning the input \ac{RNA} sequence $\phi$ from left to right,  maintaining a \textit{stack} along the way and performing one of the three actions (\textit{push, skip} or \textit{pop}). The \textit{stack} consists of a list of unpaired opening bracket positions and at each position $j = 1\dots L$, the three actions consist respectively of 1)push: opening a bracket at position $j$, 2) skip: unpaired nucleotide at position $j$ and 3) pop: closing the bracket at position $j$. Initially, \texttt{LinearFold}'s computational time was similar to the classical \ac{DP} ($O(L^3)$) because of the \textit{pop} action that involves three free indices (i.e. unpaired positions). But using a beam search heuristic, the time complexity was then reduced to $O(Lb\log  b)$, where $b$ is the beam size. The beam search is a popular heuristic technique used in computational linguistics. This technique allows keeping only the top $b$ highest-scoring (or low energy) states for each prefix of the input sequences.
\end{itemize}

%Beside the above-mentioned tools there are others score-based methods that do not implement a \ac{DP} algorithm: 
Although the score-based approaches for \ac{RNA} structure prediction often offer good accuracy and generalization,  the non-availability of the thermodynamic energy parameters for specific loops of extended sizes presents the main challenge for predicting long sequences (i.e. $L \geq 1,000$ nucleotides).  Early \ac{ML}-based methods aim to improve the energy parameters by learning the underlying folding patterns from a more considerable amount of training data.  In the next section of this chapter, we will present some of the recent improvements in structure prediction using \ac{ML}-based methods.
\subsection{\ac{ML}-based methods}
In the previous section, we reviewed the score-based \ac{RNA} secondary structure prediction methods in general and four tools in particular, i.e. \texttt{Unfold}, \texttt{RNAstructure}, \texttt{RNAfold}, and \texttt{LinearFold}. These methods are thermodynamic methods that usually rely on experimentally energy parameters. For example, most experimental energy parameters are available only for short \ac{RNA} sequences (e.g. with a length of fewer than 200 nucleotides). This limitation significantly degrades the prediction performance of thermodynamic methods for long RNA sequences. In an attempt to improve these methods, \ac{ML} methods have been proposed. This section presents an overview of existing \ac{ML} methods, especially those used in \autoref{ch:rafft} for benchmark comparison with our proposed method.

The \ac{ML}-based methods for \ac{RNA} secondary structure prediction can generally be classified into three categories according to \ac{ML}'s subprocess, i.e., score scheme based on \ac{ML}, preprocessing and postprocessing based on \ac{ML}, and prediction process on \ac{ML}. All the \ac{ML}-based methods in these three categories trained their models in a supervised way \cite{zhao2021review}. 

When using a scoring scheme based on \ac{ML},  the parameter estimation in the scoring scheme is first optimized using an \ac{ML} model. The estimated parameters are then used to evaluate the scores of possible conformations. Difference scoring schemes can be refined by using that approach: the free energy parameters, weights, and probabilities. The free energy parameter-refining is the most popular because several thermodynamic parameters of the \ac{NN} model have to be based on a large number of optimal melting experiments and the experiments are time and labour-consuming. In fact, not all free energy changes in structural elements can be experimentally measured because of technical difficulties. Instead of refining the free energy parameters, some \ac{ML}-based approaches scream through existing data of \ac{RNA} structures to extract weights that consist of different features of \ac{RNA} structure elements. These weights can be used as a scoring function for \ac{DP} techniques. The advantage of such a scoring function is that it decouples structure prediction and energy estimation. However, learned weights have no explanations because of the \ac{ML} black box. 

Another alternative for predicting \ac{RNA} structures is the \ac{SCFG} \cite{sakakibara1994stochastic, rivas2012range, dowell2004evaluation, knudsen1999rna, knudsen2003pfold, woodson2000recent}. \acp{SCFG} allow building grammar rules and induce a join probability distribution over possible \ac{RNA} structure for a given sequence $\phi$. In addition, the \ac{SCFG} models specify probability parameters for each production rule in the grammar, which allow assigning a probability to each sequence generated by the grammar. These probability parameters are learned from datasets of \ac{RNA} sequences associated with known secondary structures without carrying any external laboratory experiments \cite{dowell2004evaluation}. 

Besides the \ac{ML}-based methods that focus on refining the folding parameters, there are preprocessing and post-processing based on \ac{ML} \cite{hor2013tool, zhu2018research,haynes2008using} and direct predicting process based on \ac{ML} \cite{takefuji1990parallel,liu2006hopfield,steeg1993neural}. Preprocessing and postprocessing models allow for choosing the appropriate prediction method or set of prediction parameter sets and provide a means of determining the most likely structures among the possible outcomes that are useful for decision. The preprocessing and postprocessing \ac{ML} tools are often based on a \ac{SVM}. 

Finally, it is possible to use \ac{ML} techniques to predict \ac{RNA} secondary structure directly or combine it with other algorithms in an end-to-end fashion. Below are some of the most used and recent \ac{ML}-based tools for \ac{RNA} secondary structure prediction.
\begin{itemize}
	\item \texttt{ContraFold}\cite{do2006contrafold}: Using the so-called probabilistic model, the \ac{CLLM}, \texttt{ContraFold} appeared for the first time in early 2006. It was the first probabilistic prediction tool outperforming the existing tools, including thermodynamic tools such as \texttt{RNAfold} and \texttt{mfold}. The \ac{CLLM} is a flexible class of probabilistic models that generalizes upon \acp{SCFG}, using discriminative training and feature-rich scoring. The tool implements a \ac{CLLM} incorporating most of the features found in typical thermodynamic models allowing the tool to achieve the highest single sequence prediction accuracy to date when compared with the currently available probabilistic models.
	\item \texttt{ContextFold} \cite{zakov2011rich}: In contrast to \texttt{ContraFold}, \texttt{ContextFold} utilizes a weighted approach based on \ac{ML}. In particular, it uses a discriminative structured-prediction learning framework combined with an online learning algorithm. \texttt{ContextFold} uses a large training dataset of \ac{RNA} sequences annotated with their corresponding structures to obtain an \ac{ML} model made of $70,000$ free parameters, which has several orders of magnitudes compared to traditional models (i.e. thermodynamic free energy parameters). At its first apparition, \texttt{ContextFold}'s model succeeded at the error reduction of about $50\%$. Still, some overfitting concerns have been reported when using the tool, especially for predicting structures with large unpaired regions.  
	\item \texttt{Mxfold2}  \cite{sato2021rna}: It is one of the most recent \ac{ML}-based tools for predicting the secondary structure of \ac{RNA} molecules. Its particularity is the \ac{ML} technique used, a \ac{ML} it also belongs to the weighted approach based on \ac{ML} since the resulting model of a \ac{DNN} is a set of weight parameters. \texttt{MxFold2}'s \ac{DNN} uses the max-margin framework with thermodynamic regularization. It made the folding scores predicted by \texttt{Mxfold2} and the free energy calculated by the thermodynamic parameters as close as possible. This method has shown robust prediction on both sequences and families of natural \acp{RNA}, suggesting that the weighted \ac{ML} approaches can compensate for the gaps in the thermodynamic parameter approaches. 
\end{itemize}

Although \ac{ML} methods provide substantial improvements compared to traditional methods such as thermodynamic and comparative sequence analysis \cite{singh19_rna_secon_struc_predic_using,sato20_rna}, they often lack physical principles (training data are mostly obtained through phylogenetic analyses) and present some over-fitting concerns \cite{rivas11_range_compl_probab_model_rna}. In addition to the over-fitting problems partially due to few data availability,  \ac{ML} methods do not provide dynamic information on \ac{RNA} folding for the same reason. In \autoref{ch:rafft}, we will introduce our approach that aims at predicting an ensemble structure, which allows us to derive some dynamic information and contrasts the methods previously presented. 

\subsection{Prediction tools for pseudoknotted \ac{RNA} sequences}
In the introduction, we have provided the importance of pseudoknot interaction in realizing biological functions, and different pseudoknot patterns have been reviewed. This section introduces a couple of tools for predicting \ac{RNA} pseudoknotted structures that will be used in the benchmark results presented in \autoref{ch:arnaque}.

Folding \ac{RNA} sequences with pseudoknotted interactions is computationally more expensive than a pseudoknot-free target. Specifically, the time complexity of the pseudoknot-free secondary structure prediction is $O(L^3)$ when using dynamic programming approaches such as \texttt{RNAfold},  or less with heuristic folding methods (e.g. $O(L)$ for \texttt{LinearFold} and $O(L^2\log L)$). By contrast, when considering a special class of pseudoknots, the time complexity of folding goes up to $O(L^6)$ for an exact thermodynamic prediction using a dynamic programming approach such as \cite{pseudoknotDP}. When Using heuristic methods, the time complexity slows down to $O(L^4)$ (e.g. tools such as \texttt{IPknot} and \texttt{HotKnots}) or $O(L^3)$ for tool such as \texttt{HFold}.  
\begin{itemize}
	\item \texttt{pKiss} \cite{jangie2015}: The program \texttt{pKiss} appears the first time in 2014 as an updated version of the program \texttt{pknotsRG}\cite{reeder2007pknotsrg} which is a module of the \ac{RNA} abstract shapes analysis  \texttt{RNAshapes} \cite{jangie2015}. Initially, the program \texttt{pknotsRG} was built for the prediction of some special class of pseudoknots (unknotted structures and H-type pseudoknots). Later on, it was extended to predict \ac{RNA} structures that exhibit kissing hairpin motifs in an arbitrarily nested fashion, requiring $O(L^4)$ time. In addition to predicting the kissing hairpin motifs, \texttt{pKiss} also provides new features such as shape analysis, computation of probabilities, different folding strategies and different dangling base models. 
	
	\item \texttt{IPknot \cite{sato2011ipknot}}: it was first introduced in a paper by Kengo and his collaborators in 2011 as a novel computational tool for predicting \ac{RNA} secondary structure with pseudoknots using integer programming technique. \texttt{IPknot} uses the \ac{MEA} as a scoring function, and the maximizing expected accuracy problem is solved using integer programming with threshold cut. \texttt{IPknot} decomposes a pseudoknotted structure into a set of pseudoknot-free substructures and approximates a base-pairing probability distribution that considers pseudoknots, leading to the capability of modelling a comprehensive class of pseudoknots and running quite fast. In addition to single sequence analysis, \texttt{IPknot} can also predict the consensus secondary structure with pseudoknots when a multiple sequence alignment is given.
	
	\item \texttt{HotKnots} \cite{ren2005hotknots}. In contrast to the previously mentioned tools, \texttt{HotKnots} implements a heuristic algorithm based on the simple idea of iteratively forming stable stems. The algorithm explores many alternative secondary structures using a free energy minimization for pseudoknot-free secondary structures. Several other additions of a single substructure are considered for each structure formed at each step, resulting in a tree of candidate structures. The criterion for determining which substructures to add to partially formed structures at successive levels of the tree was also new. Similar to previous algorithms, energetically favourable substructures called \textit{hotspots} are found by a call to Zuker’s algorithm, with the constraint that no base already paired may be in the structure. 
\end{itemize}

Despite the higher computational complexity of pseudoknots, it is still important to account for them as they occur in natural \ac{RNA} and are relevant for \ac{RNA} function. We have reviewed three mainly used \ac{RNA} secondary prediction tools (\texttt{pKiss}, \texttt{IPknot}, \texttt{HotKnots}) that support the two pseudoknot patterns (i.e. the H-type and K-type) considered in \autoref{ch:arnaque}. In addition to the computational complexity, existing methods lack experimentally measured energy parameters for pseudoknot interactions. Therefore, they mostly rely or do not on approximated energy parameters, which may influence the predictions. Only \texttt{IPknot} and \texttt{HotKnots} will be used among these tools when designing pseudoknotted \ac{RNA} structures. \texttt{HotKnots} predicts the free energy of pseudoknotted structure based on recently updated energy parameters, whereas \texttt{IPknot} does not.  

So far, we have presented tools that predict a single stable and static \ac{RNA} secondary structure for a given \ac{RNA} sequence, including pseudoknots or not. More often than not, the \ac{ncRNA} functions are associated with the \acp{RNA}’ ability to undergo specific conformational changes, as is the case for riboswitches. The function of an \ac{RNA} molecule thus is usually poorly described by its ground state structure and instead has to be studied as a dynamic ensemble of structures \cite{onoa2004rna, dirks2004paradigms}. The following section will review some computational methods that address the folding dynamics of \ac{RNA} molecules.

\section{RNA kinetics} % \ensuremath{\NoCaseChange{\mathbb{ZNR}}}
The previous section introduced how pseudoknot-free secondary structures with their thermodynamic properties can be predicted. It also introduced some statistical methods that do not only rely on the thermodynamic principle but training data obtained from phylogenetic analysis, mainly the \ac{ML} methods. However, the methods used for predictions do not tell us anything about how the structures change over time and how they are related to each other. This section discusses the folding dynamics of \ac{RNA} molecules. 

The folding of \ac{RNA} molecules is remarkably more complex. It is a result of the delicate balance between multiple factors: the chain entropy, ion-mediated electrostatic interactions and solvation effect, base-pairing and stacking, and other non-canonical interactions \cite{chen2008rna}. It is a dynamic process governed by a constant formation or dissolving of base-pairs. In other terms, the \ac{RNA} molecule navigates its structure space by following a free energy landscape. Here, the free energy landscape is a high-dimensional space of all possible secondary structures ($\Sigma_{\phi}$) weighted by their free energy $\Delta G$.  


As usually done, the kinetics is modelled as a continuous-time Markov chain \cite{lorenz20_effic_comput_base_probab_multi_rna_foldin}, where populations of structure evolve according to transition rates. In this context, an Arrhenius formulation is commonly used to derive elementary transition from state $i$ to state $j$. We write 

\begin{equation}
\label{Eq:arrhenius}
k_{i \rightarrow j} = k_0 \text{exp}(-\beta  \Delta G^{\ddagger}_{i\rightarrow j})
\end{equation}
where \(\Delta G^{\ddagger}_{i \rightarrow j}\) is the activation barrier separating \(i\) from \(j\), and \(\beta=1/k_BT\) is the inverse thermal energy (mol/kcal). 
%In contrast, our kinetic ansatz uses transition rates \(r(x\rightarrow y)\) based on the Metropolis scheme already used in \cite{klemm2008funnels}, and defined as
%\begin{equation}
%\label{Eq:metropolis}
%r(x\rightarrow y) = k_0 \times \text{min}(1, \text{exp}(-\beta \Delta \Delta G(x\rightarrow y))),
%\end{equation}
%where \(\Delta \Delta G(x\rightarrow y)\) is the stability change between structure \(x\) and \(y\). 
Here \(k_0\) is the actual rate constant, solvent-dependent. Three rate models describing elementary steps in the structure space are often used to study \ac{RNA} folding dynamics: 
\begin{enumerate}
	\item The base stack model \cite{zhang02_rna_hairp_foldin_kinet,zhang2003analyzing,zhang2006exploring}: it uses base stacks as elementary kinetic move. A move consists of an addition or a breaking of a base stack with \(\Delta G^{\ddagger}_{i \rightarrow j}\) equal to the change in the entropic free energy $T\Delta S$ and the enthalpy $\Delta H$, respectively.  
	\item The base-pair model \cite{flamm2000rna,cocco2003slow}: it uses base-pair as elementary kinetic steps which gives the finest resolution, but at the cost of computation time. Here \(\Delta G^{\ddagger}_{i \rightarrow j}= \Delta G/2\) where $\Delta G$ is the energy change from state $i$ to state $j$ or $ \Delta G^{\ddagger}_{i \rightarrow j}= \Delta G$ for $\Delta G \geq 0$.
	\item  The helix stem model \cite{martinez84_rna_foldin_rule, isambert2000modeling}: the elementary move is the creation or deletion of a helix stem. It provides a coarse-grained description of the dynamics where free energy changes (\(\Delta G^{\ddagger}_{i \rightarrow j}\)) due to stem formation guiding the folding process. 
\end{enumerate}

The different rate models can lead to different folding pathways. The key factor that distinguishes the different rate models is whether the barrier is determined by ($\Delta H, \Delta S$) or by $\Delta G$. The  ($\Delta H, \Delta S$) values for different \ac{RNA} base stacks show well-separated discrete hierarchies, whereas the $\Delta G$ values show no such large separation. For two typical base stacks, 5′AU-AU3′ and 5′UC-GA3', the difference $ \Delta (\Delta H_{stact}, \Delta S_{stack}) = (7.4 \text{ kcal/mol}$, $20 \text{ kcal/mol})$ is much larger than the difference  $\Delta (\Delta G_{stack})=1.4 \text{kcal/mol}$\cite{serra199511}. Because of this fact, different models can give different folding kinetics.

Depending on the rate model used, the following master-equation describes the population kinetics $p_i(t)$ for the $i^{th}$ state ($i=1\dots \Omega$, where $\Omega$ is the total number of chain conformations)

\begin{equation}
\label{Eq:kenetics}
\frac{\text{d}p_i(t)}{\text{d}t} = \sum\limits_{j \in \Omega}
k_{j \rightarrow i} p_{j}(t) - k_{i \rightarrow j} p_{i}(t)
\end{equation}
where $k_{j\rightarrow i}$ and $k_{i \rightarrow j}$ are the rate constants for the respective transitions. The equivalent matrix form of \autoref{Eq:kenetics} is given by
\begin{equation}
\label{Eq:matrixkinetic}
\frac{\text{d}\textbf{p}(t)}{\text{d}t} = \textbf{M} . \textbf{p}
\end{equation}
where $\textbf{p} = (p_i, \dots p_{\Omega})$ is a column vector representing the frequency of structure at state $(i, \dots , \Omega)$ and, \textbf{M} is the rate matrix defined as

\begin{equation}
\textbf{M}_{ij} =  \begin{cases}
k_{i\rightarrow j},& \text{if } i\neq j\\
- \sum_{j\neq i}{k_{ij}},              & \text{if } i=j.
\end{cases}
\end{equation}

For a given initial folding condition $p_i(0)$, the \autoref{Eq:matrixkinetic} is solvable by diagonalizing the rate matrix \textbf{M} and, the solution is the population kinetics $\textbf{p}(t)$ for $t>0$ is given by

\begin{equation}
\label{Eq:solutionkinetics}
\textbf{p}(t) = \sum_{m=1}^{\Omega} {C_m \textbf{n}_m \exp{-\lambda_m t}}
\end{equation}
where $-\lambda$ and $\textbf{n}_m$ are the $m^{th}$ eigenvalue and eigenvector of the rate matrix \textbf{M}, and $C_m$ is the coefficient that is dependent on the initial condition. The eigenvalue spectrum gives the rates of the kinetic modes of the system.

Simulating the \ac{RNA} dynamics using \autoref{Eq:kenetics} has some limitations. The solution to the master-equation given by \autoref{Eq:solutionkinetics} can only give ensemble-average macroscopic kinetics and cannot give detailed information about the microscopic pathways \cite{zhang06_explor_compl_foldin_kinet_rna_hairp}. Moreover, the number of structures ($\Omega$) increases rapidly with the \ac{RNA} sequence length $L$. Therefore, the master equation is often limited to short \ac{RNA} sequences. Because of these limitations, kinetics-cluster methods are alternatively used. The basic idea of the kinetic-cluster method is to classify the large structural ensemble into a much-reduced system of clusters (of macrostates) such that the inter-cluster transitions can represent the overall kinetics. Although both the master-equation and the kinetic-cluster methods can predict the macroscopic kinetics, the kinetic-cluster approach has the unique advantage of providing direct information on the microscopic pathway statistics from the inter-cluster transitions \cite{zhang06_explor_compl_foldin_kinet_rna_hairp}. Both approaches are based on the complete conformational ensemble. An alternative approach, implemented in \texttt{kinwalker} \cite{geis2008folding}, used the observation that folded intermediates are generally locally optimal conformations.  Like thermodynamic methods for static \ac{RNA} secondary structure prediction, experimental studies usually play an essential role in guiding computational methods in studying \ac{RNA} folding dynamics.  Several recent observations are discussed in the following paragraph.
%For example, suggested quantities of biological interest include folding times, lifetimes of meta-stable states, and folding pathways, and they can be derived from the folding landscapes. 

In folding experiments, Pan and coworkers observed two kinds of pathways in the free energy landscape of a natural ribozyme \cite{pan97_foldin_rna_invol_paral_pathw}. Firstly, the investigations revealed fast-folding pathways, in which a subpopulation of \acp{RNA} folded rapidly into the native state. However, the second population quickly reached metastable misfolded states, then slowly folded into the native structure. In some cases, these metastable states are functional. These phenomena are direct consequences of the rugged nature of the \ac{RNA} folding landscape \cite{solomatin10_multip_nativ_states_reveal_persis}. 

The experiments performed by Russell and coworkers also revealed the presence of multiple deep channels separated by high energy barriers on the folding landscape, leading to fast and slow folding pathways \cite{russell2002exploring}. The formal description of the above mechanism, called the kinetic partitioning mechanism, was first introduced by Guo and Thirumalai in the context of protein folding \cite{guo95_kinet_protein_foldin}. These metastable conformations constitute competing attraction basins in the free energy landscape where \ac{RNA} molecules are temporarily trapped. However, \textit{in vivo}, folding into the native states can be promoted by molecular chaperones \cite{chakrabarti2017molecular}, which means that the active structure depends on factors other than the sequence. This may raise some discrepancies when comparing thermodynamic modelling to actual data. 

The experimental verification of the rate model is also a challenge because the microscopic elementary processes are hidden in the ensemble averages of the measured kinetics. Many researchers believe that single-molecule experiments may provide a discerning measure with careful extrapolation to the force-free case. All atom-simulations with a reliable force field and sampling method are highly valuable for providing detailed atomistic configurations for the transition state \cite{chen2008rna}. Alternatively, systematic theory-experiment tests as done in \cite{zhang06_explor_compl_foldin_kinet_rna_hairp} for designed sequences can also provide critical assessment for the different rate models.


In sum, studying the folding of RNA molecules as a dynamic ensemble of structures is of central importance in describing their functions, and experimental observations often guide the computational methods. Some of the recent experimental observations have been reviewed in this section. Among them, the kinetic partitioning mechanism is of interest in this work. It revealed the presence of multiple deep channels separated by high-energy barriers on the folding landscape, which leads to fast and slow folding pathways. The folding tool we suggest in \autoref{ch:rafft} is inspired by this mechanism and predicts fat \ac{RNA} folding pathways. The predicted pathways, therefore, allow us to derive dynamic information on RNA folding.

\section{Conclusion}
In this chapter, we have presented the \ac{RNA} folding in two main steps: (1) the prediction of the secondary structure of \ac{RNA}, which represents the static part of the folding process; (2) the \ac{RNA} kinetics, which aim at modelling the dynamics of the folding. \ac{RNA} secondary structure prediction was introduced as an optimization problem, and a review of existing methods and tools was presented. Of particular importance in this thesis's context is that existing tools for predicting \ac{RNA} secondary structures often present some limits in computational time for longer \ac{RNA} sequences. Mainly the existing tools do not give dynamical information, as few data are available on structural dynamics. Simulating the folding kinetics of long \ac{RNA} molecules is also of an essential limit because it requires a full enumeration of the structural space in most cases. In the next chapter, we will present our thesis's first result, which aims to predict \ac{RNA} folding pathways efficiently using the \ac{FFT}. The predicted pathways allow us to derive energetically suboptimal structures from which we model the \ac{RNA} folding kinetics with fewer secondary structures.

%*****************************************
%*****************************************
%*****************************************
%*****************************************
%*****************************************
