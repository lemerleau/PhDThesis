%*****************************************
\chapter{RNA folding}\label{ch:folding}

\textcolor{red}{TODO}: Provide here a short intro for the chapter 

\section{From RNA sequences to RNA structures}

This section will be dedicated at describing the folding hierarchy of RNAs. 

\begin{itemize}
	\item Different ways of representing an RNA secondary structure. 
	\item Provide an explanation on how the RNA sequences and secondary structures  are related together. Neutrality by number of structures that can take a given sequence.
\end{itemize}
\textcolor{red}{Examples: Another interesting measure in this context is the number of different RNA sequences that can fold into the a given secondary structure. This however requires some knowledge about the inverse folding problem which will be discussed in the next chapter....}


\section{Stability and prediction of RNA secondary structures}
\graffito{Note: The content of this chapter is just some dummy text.
It is not a real language.}

State different ways to predict a stable secondary structure for a given target. e.g. computationally (In \textit{silico}) and experimentally (In \textit{vitro})

\begin{itemize}
	\item The decomposition of RNA secondary structure into loops
	\item Explain the thermodynamic stability of a secondary structure and how it's computed. 
	\item  State how the MFE structure is computed: (partition function ( S. McCaskill. The equilibrium partition function and base pair binding probabilities for RNA secondary structure.; Probability etc..  Nussinov etc... ), )
\end{itemize}

\section{Energy landscape and RNA kinetics} % \ensuremath{\NoCaseChange{\mathbb{ZNR}}}

This section aim to defining what the RNA energy landscape is, and different ways of representing it. 

The second part will be for: 
\graffito{Note: For example a pic of the energy landscape of the bistable RNA.}
\begin{itemize}
	\item the partition mechanism 
	\item the kinetics of RNA 
	\item a short state of art on the kinetic tools. 
\end{itemize}

\section{A literature review of RNA folding tools.}
This section describes in details some of the existing RNA folding tools (Static folding and dynamics): 

\subsection{Exact MFE prediction methods}
\begin{itemize}
	\item \texttt{RNAfold} 
	\item \texttt{ContraFold}
	\item \texttt{RNAStructure}
	\item \texttt{LinearFold}
	\item \texttt{pKiss}
	\item \texttt{RNAExplorer}
	\item etc..
\end{itemize}

\subsection{Statical methods}
\begin{itemize}
	\item \texttt{Mxfold} 
	\item \texttt{ContextFold}
	\item etc...
\end{itemize}

\subsection{Heuristic methods}
\begin{itemize}
	\item \texttt{IPknot}
	\item \texttt{Hotknots}
\end{itemize}


\section{Conclusion}
Here will be a short conclusion of the chapter: 
%*****************************************
%*****************************************
%*****************************************
%*****************************************
%*****************************************
