%************************************************
\chapter{RNA design}\label{ch:mathtest} % $\mathbb{ZNR}$
%************************************************
\textcolor{red}{TODO: Here will be a short introduction to the chapter.}

\section{Biological motivation and biotechnological implications}
This aim to provide a biological motivation and the biotechnological implications of the RNA design  
\graffito{You might get unexpected results using math in chapter or
section heads. Consider the \texttt{pdfspacing} option.}

\section{Positive and negative design.}
\textcolor{red}{List of tasks to do}
\begin{itemize}
	\item Define the positive and negative design. 
	\item Highlight the difference between the two types of designs.  
	\item Provide a formal definition of the type of design on which this thesis is focused (RNA inverse folding). 
\end{itemize}


\section{Objective functions previously used.}
this section will provide most of the objective functions used in the RNA design

\section{A review on existing inverse RNA folding tools.}

\subsection{Pseudoknot-free RNA inverse folding tools}
\begin{itemize}
	\item \texttt{RNAinverse} 
	\item \texttt{RNAPong}
	\item \texttt{SendRNA}
	\item \texttt{ERD} (Evolutionary RNA Design)
	ERD or Evolutionary RNA Design (Esmaili-Taheri 2015)  is a recent program, first developed in 2014 (Esmaili-Taheri 2014) and one year after, an updated version has been released.  It starts by decomposing the target structure into structural components (generally called loops) and then,  independently uses an evolutionary algorithm to minimize each corresponding sub-sequence energy to recombine the different fragments to form the designed sequence finally. The main lines of ERD are: 
	Pool reconstruction: using a collection of RNA sequences (STRAN database) similar to the natural ones, a pool of sequence is constructed with respect to their length by successively finding the corresponding structure using ViennaRNA, decomposing the structure in sub-components, and finally the corresponding sub-sequences of the same length are gathered to form a pool.
	Hierarchical decomposition of the target structure into loops: using the idea that any secondary structure can be uniquely decomposed into its structural components (stems, hairpin loops, internal loops, bulge and multi-loops), ERD decomposes the target in the positions where multi-loops occur.
	Sequence initialisation: after decomposing the target structure in sub-components,  for each sub-component, a random sub-sequence is chosen from the pool and the initial sequence is a combination of those sub-sequences. 
	Evolutionary optimization of the sub-sequences: to improve the initial sequence, an evolutionary algorithm is performed on each sub-components, and the outcome sub-sequences are combined to form a newer sequence that will replace the initial one.  Iteratively the evolutionary algorithm is performed on the updated sequence until the combined sequence folds into the target or in a failure case when the stopping condition is satisfied. Two evolutionary operators are implemented here, a mutation that consists of replacing a sub-sequence corresponding to a sub-component by a new random one from the pool with respect to the length, and a selection which consists of choosing from a population of 15 RNA sequences or sub-sequences, 3 best sequences with respect to their free energy and adding them to the best from the preview generation, 3 best ones with respect to the Hamming distance from the target are therefore chosen.  The next-generation population is then obtained by generating for each of the three best sequences 5 new sequences.
	\item \texttt{MODENA}
	\texttt{MODENA} or Multi-objective Design of Nucleic Acids (Taneda 2011) is a multi-objective genetic algorithm that explores the approximate set of weak Pareto optimal solution in the space of two objective functions: one that measures the structure stability and another one that measures the similarities between the predicted secondary structure of the designed sequence and the target in order to the dominant solution.  More precisely, let 
	
	\item \texttt{NEMO \cite{nemo2018}}
	\item \texttt{SentRNA} \cite{shi2018sentrna}
	\item \texttt{antaRNA}
	\item \texttt{RNAinverse}
	\item \texttt{NUPACK}
	\item \texttt{RNAiFold}
	\item \texttt{IncaRNAtion}
	\item \texttt{INFO-RNA} 
	\item \texttt{Frnakenstein}
	\item \texttt{RNAfbinv}
	\item \texttt{RNA-SSD}
	\item \texttt{DSS-Opt}
	\item \texttt{LeaRNA}
\end{itemize}

\subsection{Pseudoknotted RNA inverse folding tools}
\begin{itemize}
	\item \texttt{Inv}
	\item \texttt{antaRNA}
	\item \texttt{MODENA}
\end{itemize}
\section{Conclusion}
Provide here a short conclusion of the chapter. 
%*****************************************
%*****************************************
%*****************************************
%*****************************************
%*****************************************
