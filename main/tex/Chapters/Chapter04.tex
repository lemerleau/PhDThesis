%************************************************
\chapter{Experimental setups and benchmark data}\label{ch:introduction}
%************************************************
This chapter aim to provide a detailed description of the benchmark methodology of folding and inverse folding tools presented in the previous chapters. 

\section{Folding tools  benchmark}\label{sec:folding_bench}

\subsection{ Benchmark datasets}
Provide here the different datasets used to benchmark the folding tools and their descriptions. 
\marginpar{\dots or your supervisor might use the margins for some
	comments of her own while reading.}
\begin{itemize}
	\item Data 1
	
	\item Data 2
	
	\item etc.. 
\end{itemize}
\subsection{Benchmark methodology }
For each entry in the above datasets we do.... 
	
	\marginpar{Options are enabled via \texttt{option=true}}
	
	\begin{itemize}
		\item:
		
		\item
		
		\item
	\end{itemize}
	


\section{Inverse folding tools  benchmark}\label{sec:folding_bench}

\subsection{ Benchmark datasets}
To estimate our EA's performance, we use two different benchmark datasets and compare the results to several existing algorithms. The two datasets were recently~used by the deep learning tool
\(\texttt{SentRNA}\) \cite{shi2018sentrna} and most of the algorithms we mentioned earlier:

\begin{itemize}
	%\tightlist
	\item The \texttt{non-EteRNA} dataset: a set of \(63\) experimentally synthesized targets that
	Garcia-Martin et al. \citep{garcia2013rnaifold} recently used to benchmark a set of ten inverse folding algorithms, which from our knowledge, is the most recent and comprehensive benchmark of current state-of-the-art methods. The dataset is collected from~\(3\)~sources: the first dataset called~\(\textbf{dataset A}\)~which contains~\(29\)~targets collected from Rfam and also used in~\citep{esmaili2015erd,taneda2011modena}~and the second called~\(\textbf{dataset B}\)~is a collection of~\(24\)~targets used in~\cite{esmaili2015erd} and added to that the~\(10\) structures used in \citep{shi2018sentrna}.
	
	\item The~\(\texttt{Eterna100}\) dataset \cite{Eterna} is available in two versions and both contain a set of \(100\) target structures extracted from the \texttt{EteRNA} puzzle game and classified by their degree of difficulty. The \texttt{Eterna100-V1} was initially designed using \texttt{ViennaRNA} 1.8.5, which relies on Turner1999 energy parameters \cite{Turn1999}. Out of the $100$ targets secondary structures, $19$ turned out to be unsolvable using the recent version of \texttt{ViennaRNA} (Version $2.14$). Subsequently, an \texttt{Eterna100-V2} \cite{Eterna} was released in which the $19$ targets were slightly modified to be solvable using \texttt{ViennaRNA 2.14}. 
	
	\item The \texttt{PseudoBase++} is a set of $265$ pseudoknotted RNA structures used to benchmark \texttt{Modena}. It was initially $304$ RNA secondary structures, but we excluded $37$ because they had non-canonical base pairs. We then grouped the structures into four pseudoknot motifs (Figure \ref{Fig:pk_type}): $209$ hairpin pseudoknots (H), $29$ bulge pseudoknots (B), $8$ complex hairpin pseudoknots (cH) and $4$ kissing hairpin pseudoknots (K).  
	

\end{itemize}

\subsection{Benchmark methodology }
For each entry in the above datasets we do.... 

\marginpar{Options are enabled via \texttt{option=true}}

\begin{itemize}
	\item:
	
	\item
	
	\item
\end{itemize}



\section{Conclusion}\label{sec:custom}
This section will provide a short conclusion for this chapter as well
