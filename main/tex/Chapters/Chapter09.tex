%************************************************
\chapter{General conclusion}\label{ch:conclusion}

This thesis has explored computational methods for studying \ac{RNA} folding. In particular, it focused on the secondary structure level. It examined the energetic and thermodynamic stability characteristics in predicting folding pathways and designing \ac{RNA} target structures through inverse folding. The principal output of the thesis is the development of computational tools to efficiently predict \ac{RNA} folding pathways using the \ac{FFT} (\texttt{RAFFT}) and an evolutionary algorithm allowing search at both local and long-range scales in the design of target \ac{RNA} structures (\texttt{aRNAque}). On the one hand, our first contribution in \ac{RNA} folding, \texttt{RAFFT}, offers an alternative computational framework to predict and study the \ac{RNA} kinetics for long \ac{RNA} molecules at lower computation costs than classical \ac{DP} methods. The versatility of our methods opens doors to different ranges of applications, such as \ac{RNA}-\ac{RNA} interactions and evolutionary dynamics. 

On the other hand, our \ac{RNA} inverse folding tool, \texttt{aRNAque}, offers a unified framework that combines the negative and positive \ac{RNA} design with an \ac{EA} that implements a Lévy flight mutation scheme. Our results show general and significant improvements in the design of \ac{RNA} secondary structures (especially on the pseudoknotted targets) compared to the standard evolutionary algorithm mutation scheme with a mutation parameter $\approx 1/L$, where $L$ is the sequence solution length. Introducing the Lévy flight mutation led to a greater diversity of \ac{RNA} sequence solutions and reduced the evolutionary algorithm's number of evaluations, thus improving computing time compared to the local search. Although \texttt{antaRNA} average \ac{CPU} time remains smaller, \texttt{aRNAque}’s success rate outperforms \texttt{antaRNA}. To further improve our program, we suggest using a more powerful computational architecture such as \ac{MPGA}. This type of architecture may allow solving more challenging target secondary structures.

Finally, we outlined these tools' limitations and prospects more generally in furthering our understanding of \ac{RNA} structure, function and design. We have put them into the context of evolutionary dynamics and highlighted potential applications in studying continuous transitions and plasticity in that context. We believe that our contributions can enhance our understanding of \ac{RNA} folding and find applications in the real world.
%—the extensions for future works and the implications for understanding their evolution. 

