%*******************************************************
% Abstract
%*******************************************************
%\renewcommand{\abstractname}{Abstract}
\pdfbookmark[1]{Abstract}{Abstract}
% \addcontentsline{toc}{chapter}{\tocEntry{Abstract}}
\begingroup
\let\clearpage\relax
\let\cleardoublepage\relax
\let\cleardoublepage\relax

\chapter*{Abstract }

%The understanding of biological processes at the molecular level has grown immensely since the discovery of structural conformations of DNA in the early 1953s. 

Since the discovery of the structure of \ac{DNA} in the early 1953s, and its double-chained complement of information hinting at its means of replication, biologists have recognized the strong connection between molecular structure and function. In the past two decades, there has been a surge of research on an ever-growing class of \ac{RNA} molecules that are non-coding but whose various folded structures allow a diverse array of vital functions. From the well-known splicing and modification of ribosomal \ac{RNA}, \acp{ncRNA} are now known to be intimately involved in possibly every stage of \ac{DNA} translation and protein transcription, as well as \ac{RNA} signalling and gene regulation processes.

Despite the rapid development and declining cost of modern molecular methods, they typically can only describe \ac{ncRNA}'s structural conformations \textit{in vitro}, which differ from their \textit{in vivo} counterparts. Moreover, it is estimated that only a tiny fraction of known \ac{ncRNA} has been documented experimentally, often at a high cost. There is thus a growing realization that computational methods must play a central role in the analysis of \acp{ncRNA}. Not only do computational approaches hold the promise of rapidly characterizing many \acp{ncRNA} yet to be described, but there is also the hope that by understanding the rules that determine their structure, we will gain better insight into their function and design. Many studies revealed that the \ac{ncRNA} functions are performed by high-level structures that often depend on their low-level structures, such as the secondary structure. This thesis studies the computational folding mechanism and inverse folding of \acp{ncRNA} at the secondary level. 
%On the one hand, the dogmatic statement that proteins are the only entities that can perform enzymatic functions within organisms has undergone a significant revision in the last few decades. More recent studies also revealed that DNAs could perform some catalytic reactions. In addition to proteins, specific RNA molecules, namely the non-coding RNAs, have accounted for their implications in most vital chemical reactions in living systems so far exclusive to proteins.

%On the other hand, advancements in developing sophisticated techniques for sequencing data and intensive lab experiments have led to identifying more non-coding RNAs involved in realizing many essential biological functions and their implications in many diseases. The advancements in studying RNA molecules and the current COVID-19 pandemic situation have contributed more to shifting the attention from DNA and protein to RNAs. 

%Computationally, folding an \ac{RNA} molecule to its secondary structure involves finding the configuration with the minimum free energy in the space of all possible secondary structures. In contrast, the inverse folding problem consists of searching in the space of \ac{RNA} sequences for those whose minimum free energy structures are similar to a given target structure. Addressing both problems often requires an energy function that allows mapping each \ac{RNA} molecule and a probable secondary structure to a free energy value. Such an energy function often relies on thermodynamic parameters experimentally measured. 

In this thesis, we describe the development of two bioinformatic tools that have the potential to improve our understanding of \ac{RNA} secondary structure. These tools are as follows: (1) \texttt{RAFFT} for efficient prediction of pseudoknot-free \ac{RNA} folding pathways using the \ac{FFT}; (2) \texttt{aRNAque}, an evolutionary algorithm inspired by Lévy flights for  \ac{RNA} inverse folding with or without pseudoknot (A secondary structure that often poses difficulties for bio-computational detection). 

The first tool, \texttt{RAFFT}, implements a novel heuristic to predict \ac{RNA} secondary structure formation pathways that has two components: (i) a folding algorithm and (ii) a kinetic ansatz. When considering the best prediction in the ensemble of $50$ secondary structures predicted by \texttt{RAFFT}, its performance matches the recent deep-learning-based structure prediction methods. \texttt{RAFFT} also acts as a folding kinetic ansatz, which we tested on two  \acp{RNA}: the \ac{CFSE} and a classic bi-stable sequence. In both test cases, fewer structures were required to reproduce the full kinetics, whereas known methods (such as \texttt{Treekin}) required a sample of $20,000$ structures and more. 

The second tool, \texttt{aRNAque}, implements an \ac{EA} inspired by the Lévy flight, allowing both local global search,  and which supports pseudoknotted target structures. The number of point mutations at every step of \texttt{aRNAque} \ac{EA} is drawn from a Zipf distribution. Therefore, our proposed method increases the diversity of designed \ac{RNA} sequences and reduces the average number of evaluations of the evolutionary algorithm. The overall performance showed improved empirical results compared to existing tools through intensive benchmarks on both pseudoknotted and pseudoknot-free datasets. 

In conclusion, we highlight some promising extensions of the versatile \texttt{RAFFT}'s method to \ac{RNA}-\ac{RNA} interaction studies. We also provide an outlook of both tools' implications in studying evolutionary dynamics. 
\vfill

%\begin{otherlanguage}{ngerman}
%\pdfbookmark[1]{Zusammenfassung}{Zusammenfassung}
%\chapter*{Zusammenfassung}
%Kurze Zusammenfassung des Inhaltes in deutscher Sprache\dots
%\end{otherlanguage}

\endgroup

\vfill
