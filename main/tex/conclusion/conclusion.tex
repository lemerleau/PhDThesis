%use in documents with \subfile{tex/intro}
\documentclass[../../master.tex]{subfiles}

\begin{document}

\subsection{Conclusion}
\label{sub:discussion:conclusion}

%\begin{itemize}
	%\item structure comparison: tree based expensive, ASPRAlign problematic. BP distance with PK works well enough for small distances BUT: due to rather low quality of MFE predictions
	%\item \parencite{dingle_structure_2015} so: seq.space>shapespace. given shape, there are lots of sequences. that's great; makes design easier BUT: since we want catalytic function, we still have a vast set of options to choose from and we don't know what to optimize for.
	%\item \parencite{ameta_next-generation_2014} this is interesting: catalysts evolving from random space; validates the idea to design sequences and justifies not exact designs, as we could evolve them.
%\end{itemize}


%\todo{take away: nested tooling for PK design. The general approach is not special. Lots of parameters to tweak from prediction to comparison, model, objective. search space still huge and apart from secondary structure, formulating a fitting objective for functional design is difficult but would be effective if achievable}

Within limits of the underlying model and pseudoknot heuristics, designing secondary structures including simple pseudoknots works surprisingly well by utilizing methods developed for minimum free energy prediction of nested structures in conjunction with sequence and structure constraints.
Even further, by diverting McCaskill's partition function algorithm from its intended use with nested secondary structures, structural constraints may be omitted.
However, the hope for success of the designed sequences in wet-lab experiments is questionable.
The diversity of the designed sequences indicates that the solution space is still vast.
Relating secondary structure objective functions to \textit{in vitro} experiments testing catalysis is challenging.
But there is potential in evolving sequences designed for secondary structure \textit{in vitro} as indicated by RNA shape space covering, i.e. most common secondary structures being reached from random sequences by just a few mutations \parencite{schuster_sequences_1994, reidys_generic_1997}, and experimental evidence showing improvements of catalytic activity by \textit{in vitro} evolution \parencite{ameta_next-generation_2014}.

Shifting from a purely thermodynamical view on secondary structures to kinetic models and modelling tertiary interactions seems necessary to design functional ribozymes similar to the \textit{Azoarcus} group I intron.


\end{document}
