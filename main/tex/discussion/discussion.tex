%use in documents with \subfile{tex/intro}
\documentclass[../../master.tex]{subfiles}

\begin{document}

\section{Discussion}
\label{sec:discussion}

Ultimately, the catalytic activity of the \textit{Azoarcus} group I intron determined the focus of the overall design on the pseudoknot and specifically the P7 region containing the guanosine binding site.
Therefore, two different approaches were pursued.

The first one employed structural and sequence constraints to keep the P7 region intact, motivated by the binding site relevant for catalytic activity that P7 contains, and the strong conservation of the region among group I introns.

The second, alternative approach started as a modification of the first one.
The core idea was inspired by an observation made by Gaspin and Westhof and later applied to the \textit{Tetrahymena} group I intron by Mathews; base pair probabilities obtained using the McCaskill algorithm indicate possible pseudoknots \parencite{gaspin_interactive_1995, mathews_using_2004}.
Using the normalized ensemble defect as the primary objective function required computation of the partition function, so no computational cost was added apart from using computed base pair probabilities twice.

Notably, no explicit modelling of pseudoknots was necessary for the implementation of both design approaches.
Naturally, it should be questioned how well these approaches suited the primary goal of generating RNA sequences with pseudoknots similar to the one present in the target structure.

In attempting to address this question from a technical perspective, the challenges and limitations of the design approaches implemented in this work are discussed in \autoref{sub:discussion:challenges_limitations}.
In that context, the desired catalytic activity of the designed RNA sequences in wet-lab experiments should not be ignored, and core assumptions are revisited.
Finally, improvements to the design pipeline are proposed, and various other approaches are suggested (\autoref{sub:discussion:methodological}).

\end{document}
