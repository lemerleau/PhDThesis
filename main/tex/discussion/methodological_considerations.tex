%use in documents with \subfile{tex/intro}
\documentclass[../../master.tex]{subfiles}

\begin{document}

\subsection{Methodological Considerations}
\label{sub:discussion:methodological}

Following the previous section, many possible changes could be applied to the design pipeline.
For example, a structure predicted for the reference sequence could be used as a target structure and selected sequence constraints could be reconsidered. 
Moreover, other characteristics like the free energy of predicted structures or ensemble diversities obtained from the partition function could be incorporated as part of the objective function.
These are rather general modifications that do not reach beyond the conceptual limits of the model.

Instead, let us consider specific changes motivated by four central building blocks of the design pipeline.

First, sequence constraints were used to na\"{i}vely account for tertiary interactions.
Similar to how pseudoknots are a violation of rule \ref{eq:rule3} (see \autoref{sec:theory:rna_secstructures}) allowing unconveniently many structures, tertiary interactions violate rules \ref{eq:rule1} and \ref{eq:rule4}. 
Allowing structures with nucleotides involved in multiple, possibly noncanonical pairs seems as unattractive as arbitrary kinds of pseudoknots.
For that reason, it might be more feasible to impose some restrictions and model only some tertiary interactions, similar to the approach of \texttt{RNAwolf}, modelling nucleotides forming up to two potentially noncanonical pairs \parencite{zu_siederdissen_folding_2011}.

Secondly, neutral networks percolating the sequence space are not rare even for secondary structures with pseudoknots. 
The sequences designed in this work are not some rare examples with confined neutral networks.
Inspired by \parencite{reidys_generic_1997}, random drift could be introduced in the design pipeline by allowing moves between neutral neighbors.
Therefore, the parameter $p_\mathrm{acc}$ could be repurposed or even omitted to allow neutral moves (see \autoref{par:methods:pipelineoverview}).

Thirdly, sampling sequences compatible to multiple target structures was facilitated by \texttt{RNAblueprint}.
Although used to handle a pseudoknot in this work, this library allows sampling sequences subject to more complex structural constraints. 
It has been successfully used in RNA design with multiple conformational states \parencite{findeis_silico_2018}.
With the conformational change of the \textit{Azoarcus} group I intron during self-splicing in mind, a similar approach should be explored (cf. \autoref{sub:theory:azoarcus_selfsplicing}).

Finally, the efficient partition function and base pair probability computation via McCaskill's algorithm could replace MFE prediction directly.
Nevertheless, computing a partition function for ensembles of nested structures to detect pseudoknots has limitations.
Detecting more complex or multiple pseudoknots using dot plots of base pair probabilities could be problematic because base pair probabilities at corresponding positions might be indistinguishable from alternative nested conformations.
In fact, this problem translates to the extension of the maximum normalized ensemble defect to multiple superpositioned target structures (see \autoref{eq:maxned}).
Minimizing this objective function may be interpreted as concurrent minimization of normalized ensemble defects for multiple conflicting nested target structures.
Fortunately, the ensemble defect relies indirectly on the ensemble of possible structures via the base pair probabilities obtained from the partition function.
Using crossing structures as input for the ensemble defect is merely an implementation detail.
Of course, this is usually questionable, but it makes sense in the case of this work.
%\todo{not precisely equivalent to maxned, which does implicit weighting based on superposition. the modified ned does not weight but also minimization of this will not reach zero. BUT: I think we should be able to calc. that minimal value depending on conflicting pairs?-> renorm BPPM}
For this reason, I extended the ensemble defect implementation in \texttt{ViennaRNA}. Those changes were accepted and are available in the \texttt{ViennaRNA} package as of version \texttt{2.4.18} (see \autoref{sub:appendix:code_availability}).

A more generalized version of the ensemble defect could be used in order to compare structure ensembles of sequences \parencite{dirks_paradigms_2004}:
\begin{equation}\label{eq:frobeniusbased}
	\operatorname{d}(P,P') = N - \sum_{i,j} P_{i,j} P'_{i,j}
\end{equation}
where $P$ and $P'$ are base pair probability matrices of two sequences.
The distance in \autoref{eq:frobeniusbased} may be seen as an estimation of the metric induced by the Frobenius norm (see \autoref{sub:appendix:frobenius}).
Then again, this metric is computationally more expensive than the ensemble defect.

Early on in this work, using a partition function algorithm explicitly modelling some types of pseudoknots implemented in \texttt{NUPACK} was disregarded.
However, there is a design approach implemented in \texttt{ENZYMER} similar to the alternative approach of this work; it primarily consists of an adaptive walk --- with varying step sizes --- minimizing the ensemble defect, which was computed using \texttt{NUPACK} \parencite{zandi_adaptive_2016}.

%\todo{Feels \emph{very} dual to MEA prediction (MEA prediction: fixed sequence, find structure with max. MEA vs. ED design: fixed structure, find seq. with min ED (or max. MEA))}}


\end{document}
