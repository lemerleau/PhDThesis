%use in documents with \subfile{tex/intro}
\documentclass[../../master.tex]{subfiles}

\begin{document}

\subsubsection{Outlook on Other Approaches}
\label{ssub:discussion:other_approaches}

The design approaches in this work followed a \textit{de novo} pattern by starting from an arbitrary sequence and exploring the sequence space.
A straightforward variation of the herein implemented design pipeline would utilize the reference (or hypothetical promising sequence designs) as the initial sequence.
Similarly, the neutral network of the reference sequence could be explored.
Generally, these variations would not address the limitations outlined in \autoref{sub:discussion:challenges_limitations}.

Extending on multi-state RNA design, briefly hinted at in \autoref{sub:discussion:methodological}, incorporating RNA folding kinetics in a design pipeline should be worthwhile.
Not only does RNA fold hierarchically, i.e. from secondary to tertiary structure, but also sequentially via intermediate conformational (secondary structure) states \parencite{tinoco_how_1999}.
Indeed, group I introns like the one in \textit{Azoarcus} fold into intermediate structures along conserved pathways \parencite{mitra_rna_2011}.
While there are computational models of RNA folding kinetics available \parencite{kucharik_basin_2014, kucharik_pseudoknots_2016}, modelling kinetics for every design candidate might not be feasible.
Instead, intermediate conformations computed for the reference could be used to define multiple target structures and energies.
%\todo{name basin hopping graph?}
%\todo{\parencite{gupta_identifying_2012}}

In order to enable more complex approaches, reducing the size of the reference molecule might be in order.
In the case of the \textit{Azoarcus} group I intron, removing the scaffold domain does not entirely prohibit the catalytic activity of the remaining molecule, enabling the construction of a smaller ribozyme.
With that, some new complications might arise because the scaffold domain contributes significantly to the stability of the \textit{Azoarcus} group I intron and increases mutational robustness \parencite{hayden_intramolecular_2015}.
Still, the construction of a minimal synthetic intron is a way to reduce computational complexity but also requires experimental verification of catalytic activity beforehand.
Similarly, designing only fragments with the goal of self-assembly might reduce complexity while simultaneously complicating experimental verification.


Another direction could be taken by utilizing phylogenetic data, e.g. from \texttt{GISSD} \parencite{zhou_gissd_2008}.
In fact, the unlabelled lanes in \autoref{fig:gel_proto} represent sequences designed by Vaitea Opuu based on \textit{Azoarcus} group I intron homologs.
One of his approaches utilized \emph{Direct-Coupling Analysis} (DCA).
The core idea of DCA is to model the probability of a sequence parametrized by coupled nucleotides and position-specific biases for specific nucleotides \parencite{de_leonardis_direct-coupling_2015}.
With parameters derived from multiple sequence alignments of homologous sequences, DCA can be used for structure prediction, including tertiary interactions, and sequence design by maximizing probability according to the model.


%outlook: tert. interactions: alignment-based statistical methods?
%lattice based methods for structure prediction -> not limited to sec. structure?



Yet, all of these suggestions represent computational methods, and their validity strongly depends on experimental verification.

\end{document}
