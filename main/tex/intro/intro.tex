%use in documents with \subfile{tex/intro}
\documentclass[../../master.tex]{subfiles}

\begin{document}

\section{Introduction}
\label{sec:intro}

Ribonucleic acid (RNA) and desoxyribonucleic acid (DNA) are very similar biopolymers from a molecular point of view.
The role of DNA in biology is commonly known to be the storage of genetic information in all known living organisms \parencite{lodish_molecular_2000}.
%\todo{(in fact, I first learned in primary school about the DNA double helix) \emph{I'll delete this later}}
In contrast, the functions of RNA molecules are more diverse.
One of the most widely known classes of RNAs is messenger (m)RNA, which acts as an intermediate carrier of genetic information in the translation apparatus of the cell. 
It is synthesized as a complementary copy of gene-coding DNA and serves as a template for protein translation in the ribosome.
Nevertheless, large parts of the DNA do not encode for proteins.
Instead, RNA transcribed from such regions in the genome may possess very different functions.
These include ribosomal (r)RNA, executing the actual translation in the ribosome, transfer (t)RNA also participating in translation, as well as many classes of other small non-coding RNAs carrying out specific catalytic functions \parencite{hofacker_rna_2006}.

One such catalytically active RNA molecule found in the bacterium \textit{Azoarcus} is the subject of this thesis.
RNAs similar to this molecule are often assumed to have played a significant part in \emph{abiogenesis}, the emergence of life from inanimate matter or, loosely speaking, the origin of life.
The critical characteristics of molecules relevant to the origin of life revolve around self-replication, requiring to pass on heritable information and carrying out some function.
The RNA molecule from \textit{Azoarcus} is a promising candidate.

Is it possible to design similar synthetic RNA sequences?
In an attempt to sequence design using computational methods, the structure of RNA is used as a first proxy for RNA function in this work.
The following sections shall establish context around the level of abstraction at which structure is modelled in this work the significance of RNA in the origin of life.


\subsection{Biochemical Context}
\label{sub:intro:rnas}

The function of RNA is closely tied to its structure.
So, to understand RNA function, it is beneficial to understand where its structure emerges from.
The smallest unit of an RNA polymer is a nucleotide.
Common to every (monophosphate) nucleotide in RNA is its backbone consisting of ribose with a phosphate residue at its fifth carbon atom.
The identity of a nucleotide is determined by one of four nucleobases attached at the first carbon atom of the central ribose molecule; adenine (A), guanine (G), cytosine (C) and uracil (U).

Repeated binding of a nucleotide's phosphate residue to the third carbon atom of another nucleotide's ribose by replacing the hydroxy group yields sequential polynucleotides \parencite{lodish_molecular_2000}.
A so-called phosphodiester bond connects the nucleotides.
By repeatedly establishing so-called phosphodiester bonds between the phosphate residue of one nucleotide and the ribose of another, polynucleotide chains are formed \parencite{lodish_molecular_2000}.
The structure formula of the nucleotide guanosine monophosphate is depicted in \autoref{fig:gmp} as an example.

\begin{figure}[!ht]
	\centering
	\footnotesize
	\chemfig{!{GMP}}
	\caption[Structure of Guanosine Monophosphate]{
		Structure formula of guanosine monophosphate.
		\textcolor{gray}{Gray} labels denote the carbon atoms whose residues participate in phosphodiester bonds of RNA polymers. The formation of a phosphodiester bond requires a hydrogen residue from the phosphate and a hydroxy group from the ribose.
	}\label{fig:gmp}
\end{figure}

The formation of these linear, directed polynucleotides leads to the primary structure of RNA defined by the sequence of nucleotide identities.
Similar to DNA, nucleobases can form pairs via hydrogen bonds.
These pairings are specific and define a characteristic secondary structure \parencite{hofacker_rna_2005}.
The canonical base pairs of RNA secondary structure are shown in Figure \ref{fig:basepairs}.

\begin{figure}[!ht]
	\centering
	\subcaptionbox{\textbf{A}denine and \textbf{U}racil \label{fig:basepairs:a}}
	{
		\scriptsize
		\chemfig{\textbf{H}!{AU}\textbf{H}-[:-90,,,,draw=none]\phantom{A}}
	}
	\subcaptionbox{\textbf{G}uanine and \textbf{C}ytosine\label{fig:basepairs:b}}
	{
		\scriptsize
		\chemfig{\textbf{H}!{GC}\textbf{H}}
	}
	\subcaptionbox{\textbf{G}uanine and \textbf{U}racil\label{fig:basepairs:c}}
	{
		\scriptsize
		\chemfig{\textbf{H}!{GU}\textbf{H}}
	}
	
	\caption[Canonical RNA Base Pairs]{
		The canonical base pairs in RNA.
		\begin{enumerate*}[label={(\alph*)}, font={\bfseries}]
			\item and
			\item are commonly known as Watson-Crick base pairs.
			\item is usually called a wobble base pair.
		\end{enumerate*}
		Hydrogen bonds are indicated as gray dashed lines.
		Substitution of \textbf{bold hydrogen} residues with ribose-5-phosphate yields the corresponding nucleotides found in RNA polymers.
	}\label{fig:basepairs}
\end{figure}

However, the formation of other less specific, \emph{noncanonical} base pairs is possible.
Those pairings facilitate the formation of three-dimensional or tertiary structure.

In summary, the structure of an RNA molecule is hierarchical; the primary structure is defined by the sequence of nucleotides, features of the secondary structure are formed by base pairs determined by the primary structure, and at last, tertiary structure arises from non-specific contacts between nucleotides and is sterically constrained by secondary structure \parencite{tinoco_how_1999}.

The second level of the structural hierarchy is particularly interesting. 
It guides tertiary structure formation and enables efficient computational methods due to the relatively simple rules that suffice to describe how secondary structure arises from primary structure \parencite{hofacker_rna_2005}.

\end{document}
