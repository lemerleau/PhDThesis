%use in documents with \subfile{tex/intro}
\documentclass[../../master.tex]{subfiles}

\begin{document}

\subsection{The RNA World Hypothesis}
\label{sub:intro:rnaworld}

%How was the first heritable information stored and how was catalytic function achieved?
%Questions like these aim at finding paths of how a translation apparutus like in all living organisms could emerge.

The fundamental question of the origin of life is, broadly speaking, concerned with the emergence of the first cellular organisms from a primordial soup of inanimate molecules.
At a closer look, the search for an answer has to address multiple issues to bootstrap the first proto-cells.

In the most simplistic view of a cell hypothesized by \citeauthor{crick_protein_1958}, the genetic information stored in DNA is transcribed into RNA, which is translated into proteins, often carrying out catalytic activity \parencite{crick_protein_1958}.
There is a dichotomy between nucleic acids carrying genetic information and proteins carrying out function, seemingly requiring both types of biopolymers interacting in the translation apparatus to facilitate Darwinian evolution of cells.

What did the first self-replicating molecules preceding cellular organisms look like?
Such molecules would have been required to both store heritable information and carry out some catalytic function.
\citeauthor{orgel_evolution_1968} argued that either nucleic acids or proteins could be such early self-replicating molecules \parencite{orgel_evolution_1968}.
At this point, the existence of catalytically active nucleic acids was unknown, and \citeauthor{orgel_evolution_1968} was uncertain whether nucleic acid --- DNA or RNA --- chains with well-defined secondary structure would be capable of catalysis \parencite{orgel_evolution_1968}.

The view of RNAs being the first or at least early self-replicating molecules is known as the \emph{RNA world hypothesis}, a term phrased by \citeauthor{gilbert_origin_1986} after the first ribozymes catalyzing self-excision from surrounding RNA have been discovered in \textit{Escherichia coli} and \textit{Tetrahymena} \parencite{gilbert_origin_1986}.

\citeauthor{eigen_principle_1977} introduced a model of pre-cellular Darwinian systems, the \emph{hypercycle}, building a complex translation apparatus from initial self-replicating units although they assumed nucleic acids to be limited in their information storage capacity so that multiple interacting replicators would be necessary \parencite{eigen_principle_1977}.
Later, they proposed a concrete hypercycle starting from polynucleotides to bootstrap protein translation by tRNA-like self-replicating precursors and argued for RNA to precede DNA as the first genetic information-carrying molecule interacting with primitive catalytic proteins \parencite{eigen_hypercycle_1978-1, eigen_origin_1981}.

Although RNA is often seen as fundamental to the origin of life, the hypothesized RNA world is not necessarily the beginning of it all. 
RNA-like worlds preceding RNA itself have been proposed as alternatives in which ribose would have been substituted by peptides, threose, or glycol \parencite{robertson_origins_2012}.

Research efforts aiming to support the RNA world hypothesis or related concepts have taken various forms.
Early attempts by Orgel, Zielinski, Kiedrowski and Sievers were based on DNA and DNA-analogue oligomers and yielded non-enzymatic template-directed self-replicators \parencite{kiedrowski_self-replicating_1986, zielinski_autocatalytic_1987, sievers_self-replication_1994}.
Approaches taken by scientists around Joyce have focused on finding replicase ribozymes, arguing that replicase functionality is within reach from existing polymerase ribozymes \parencite{mcginness_search_2003, robertson_origins_2012}.
In fact, \citeauthor{breaker_emergence_1994} constructed a self-replicating RNA species which still required a DNA dependant RNA polymerase \parencite{breaker_emergence_1994}.

However, template-based approaches were often limited as the product often diffuses slowly away from the template and inhibits further self-replication \parencite{hayden_self-assembly_2006}.
The approach successfully pursued by \citeauthor{hayden_self-assembly_2006} built upon recombination of smaller fragments into a functional ribozyme, eliminating the need for a template and enabling both self-assembly and autocatalysis at the same time \parencite{hayden_self-assembly_2006}.

Perhaps not surprisingly, a large part of research in the RNA world context built upon ribozymes related to the RNA molecules deemed promising by \citeauthor{gilbert_origin_1986} \parencite{gilbert_origin_1986}. 
These self-splicing ribozymes often found in tRNA precursors are called group~I~introns.
The self-assembling autocatalyst Hayden used in his work belongs to this class of ribozymes and was used throughout this work as the design reference (see \autoref{sec:theory:gii}).


%This function and the reverse re-insertion was seen as evidence for early transposable elements, effectively enabling some form of recombination.

%The attribution of information storage and catalytic function to RNA molecules at an early stage in the origin of life is central to variants of the RNA world hypothesis.



%A different direction is taken by metabolism-first views, referring to metabolic networks capable of information transfer without sequence-based biopolymers \parencite{segre_compositional_2000, lancet_systems_2018}.
%The evolvability of such system remains questionable though \parencite{vasas_lack_2010}. 

%On the other hand, over the last decades evidence compatible with Eigen's quasi-species and the picture of self-splicing RNA facilitating recombination \parencite{gilbert_origin_1986} has been accumulated, ranging from building RNA replicase candidates from RNA polymerases~\parencite{mcginness_search_2003} to constructing quasi-species of self-replicating RNAs~\parencite{hayden_self-assembly_2006, lincoln_self-sustained_2009}, as well as demonstrating cooperativity~\parencite{hayden_systems_2008} mutational robustness~\parencite{hayden_intramolecular_2015} in such systems.


%\parencite{lincoln_self-sustained_2009} another autocatalyst, or rather two cooperative? 


\end{document}
