%use in documents with \subfile{tex/intro}
\documentclass[../../master.tex]{subfiles}

\begin{document}

\section{Methods}
\label{sec:methods:methods}

The methods used in this work are based on the overall goal of sequence design (see \autoref{sub:intro:outline}).
Implementations of different thermodynamical structure prediction algorithms and energy parameter sets of the underlying model 
are introduced in \autoref{sub:methods:sspred}.

In an approximation of inverse folding, the design pipeline described in detail in \autoref{sub:methods:design_pipeline} follows a general feedback loop starting from a candidate sequence whose predicted structure is compared to the target structure. 
By mutating candidate sequences and repeated structure prediction, the general design goal consists of exploring the sequence space and increasing similarity to the target \parencite{dirks_paradigms_2004}.
Often, the distinction of positive and negative RNA design is made. 
The former attempts to increase affinity, whereas the latter aims to amplify specificity for a target structure \parencite{dirks_paradigms_2004}.
Whether a design pipeline implements positive or negative design depends on the choice of concrete objective functions used to evaluate predicted structures (\autoref{ssub:methods:objectives}).

Information about the exact versions of computational tools used in this work and the availability of self-written code was appended in \autoref{sub:appendix:code_availability}.

\end{document}
