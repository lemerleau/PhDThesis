%use in documents with \subfile{tex/intro}
\documentclass[../../master.tex]{subfiles}

\begin{document}

\subsection{Secondary Structure Prediction}
\label{sub:methods:sspred}


%\todo{a methodological issue I see that needs to be addressed in the discussion: PK penalty in pKiss: 9.0 or 9.8, in RNAPKplex I used no penalty, in RNAfold I used no penalty (when calculating the contribution of P7 base pairs). This needs to be discussed or better justified. So far my intuition is: PK penalty values are not well known anyway, it's questionable if a constant penalty is adequate? pKiss uses same energy parameters but model is still different. In the end I only use RNAfold for the design and the other tools are used only for exploration. Stabilizing effect of tertiary interactions?}

A small variety of tools based on the thermodynamical model of secondary structures, as well as different sets of energy parameters, were assessed for their ability to recover large parts of the target structure, given the native sequence with a focus on its pseudoknot (see \autoref{sub:theory:seqtargetstruct}).

\subsubsection{\texttt{pKiss}}
\label{ssub:methods:pkiss}

\texttt{pKiss} implements an algorithm for the prediction of minimum free energy (MFE) structures with different heuristics to predict a few kinds of simple pseudoknots \parencite{janssen_rna_2015}.
Here, due to the simple H-type nature of the pseudoknot in the target structure, the legacy strategy only considering simple recursive pseudoknots from its predecessor \texttt{pknotsRG} was chosen \parencite{reeder_design_2004, reeder_pknotsrg_2007}.

However, the class of simple recursive pseudoknots does not include helices with \unit[1]{nt}-bulges as seen in region P7 of the \textit{Azoarcus} group I intron.
This was considered a relatively minor inaccuracy in this work and could be bypassed.
The bulge was temporarily disregarded to estimate the energy of the native structure with its sequence (see \autoref{ssub:results:azopred}).

Experimental data for the energy contribution of pseudoknots tends to be scarce, and energy approximations have some limitations \parencite{gultyaev_approximation_1999, fallmann_recent_2017}.

In \texttt{pKiss}, the H-type pseudoknot initiation cost (see \autoref{sub:theory:pseudoknots}) is assumed constant with a default energy contribution of $\unitfrac[9]{kcal}{mol}$.
In the literature, initiation cost approximations range from $\unitfrac[7]{kcal}{mol}$ \parencite{rivas_dynamic_1999} to roughly $\unitfrac[10]{kcal}{mol}$ ($\unitfrac[9.6]{kcal}{mol}$ in \parencite{dirks_partition_2003}), which is why values in this range were tested to possibly improve the prediction of the pseudoknot in the native sequence.

\subsubsection{\texttt{RNAfold}}
\label{ssub:methods:rnafold}

\texttt{RNAfold} from the \texttt{ViennaRNA} package \parencite{lorenz_viennarna_2011} was chosen for its speed and versatility despite being restricted to nested secondary structure prediction.

Structural constraints prohibiting positions in P7 from forming base pairs were applied during the prediction of minimum free energy structures to circumvent this limitation.
This approach entailed constraints on candidate sequences at those positions in the design process (see \autoref{sub:methods:design_pipeline}).
The predicted free energies were then adjusted using the energy contribution of the base pairs in P7 (see \autoref{fig:p7energy:a}).
A penalty for implicitly adding the pseudoknot via prediction constraints was disregarded since this would not change the predicted structures.

Additionally, the \texttt{ViennaRNA} package provides an implementation of the McCaskill algorithm to compute the partition function and base pair probabilities for nested secondary structures.
Using this feature enabled an alternative approach to the aforementioned constrained structure prediction; 
by examining dot plots of the base pair probabilities of an RNA sequence, potential pseudoknots can be extracted (see \autoref{sub:theory:pseudoknots}).

A partition function algorithm implementation for structures with pseudoknots from \texttt{NUPACK} was briefly used for comparison \parencite{dirks_partition_2003, dirks_paradigms_2004}.


\subsubsection{\texttt{RNAPKplex}}
\label{ssub:methods:rnapkplex}

\texttt{RNAPKplex} provides minimum free prediction of structures containing at most one pseudoknot \parencite{lorenz_viennarna_2011, beyer_rna_2010}.
The heuristic of \texttt{RNAPKplex} searches for intervals in an MFE structure that can be made accessible by removing base pairs to form a pseudoknot if the latter compensates the loss of stability by making intervals accessible.
This algorithm was chosen for comparison since the energy calculation for its predictions is relatively similar to the constrained approach using \texttt{RNAfold} albeit more general.
For this reason, no pseudoknot initiation cost was used with \texttt{RNAPKplex} after initial tests.

During work on this thesis, basic functionality of \texttt{RNAPKplex} was restored and merged into \texttt{ViennaRNA} (see \autoref{sub:appendix:code_availability}).
As of version \texttt{2.4.18}, the overall implementation was extensively rewritten by Ronny Lorenz, meaning that the results in this work are not indicative of the latest changes to \texttt{RNAPKplex}.

%\todo{How do I refer to my fixes properly? they're merged into \textbf{user-contrib} on github}
%\todo{the predictions I tested in Beyers thesis weren't reproducible with the default penalty after my fixes}.


\end{document}
