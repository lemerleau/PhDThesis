%use in documents with \subfile{tex/intro}
\documentclass[../../master.tex]{subfiles}

\begin{document}

\subsection{Native Sequence and Target Structure Data}
\label{sub:theory:seqtargetstruct}

%\todo{maybe moved into methods (but only if somebody cares enough)}
Since a comprehensive set of tertiary interactions in the \textit{Azoarcus} group I intron was compiled in \parencite{mustoe_secondary_2016}, the sequence and structure data used there served as the starting point for this work (\autoref{fig:azodata:a}).

The tertiary structure the simulations in \parencite{mustoe_secondary_2016} were based on originated from crystallography, which was facilitated by co-crystallization of an RNA binding protein and therefore required the insertion of a matching binding site into the intron sequence itself \parencite{adams_crystal_2004}.

\begin{figure}[!ht]
	\centering
	\begin{subfigure}[t]{\textwidth}
		\centering
		\ttfamily
		\begin{tabularx}{0.883\textwidth}{X}
			\hphantom{gcggacucauauu}\seqsplit{{\textcolor{black}{gccgu}}GUGCCUUGCGCCGGGAAACCACGCAAGGGAUGGUGUCAAAUUCGGCGAAACCUAAGCGCCCGCCCGGGCGUAUGGCAACGCCGAGCCAAGCUUCGCAGCC{\icbox{{AUUGCACUCC}}}GGCUGCGAUGAAGGUGUAGAGACUAGACGGCACCCACCUAAGGCAAACGCUAUGGUGAAGGCAUAGUCCAGGGAGUGGCGAAAGUCACACAAACCGG} \\
			\hphantom{gcggacucauauu}\seqsplit{{\textcolor{gray}{.....}}...(((((((..((....)).)))))))...((((((....((((((...((...((((((....))))))..))...))))))(((...(.(((((((({\icbox{\vphantom{A}{..........}}})))))))).)..)))...[.[[[[[...))))))((((..((((....))).)))))......]]]]]].((..(((((....))))).....))..} 
		\end{tabularx}
		\caption{Sequence and secondary structure as used in \parencite{mustoe_secondary_2016}.
		}\label{fig:azodata:a}
	\end{subfigure}
	\newline
	\begin{subfigure}[t]{\textwidth}
		\centering
		\ttfamily
		\begin{tabularx}{0.883\textwidth}{X}
			\seqsplit{{\textcolor{gray}{gcggacucauauuucgau}}
				GUGCCUUGCGCCGGGAAACCACGCAAGGGAUGGUGUCAAAUUCGGCGA
				AACCUAAGCGCCCGCCCGGGCGUAUGGCAACGCCGAGCCAAGCUUCGG
				CGCC{\icbox{\phantom{{AUUGCACUCC}}}}UGCGCCGAUGAA
				GGUGUAGAGACUAGACGGCACCCACCUAAGGCAAACGCUAUGGUGAAG
				GCAUAGUCCAGGGAGUGGCGAAAGUCACACAAACCGG
				{\textcolor{gray}{aauccguugg}}} \\
			\seqsplit{{\textcolor{gray}{.......(((......))}}
				{\stackengine{0pt}{\textcolor{gray}{)}}{.}{O}{c}{F}{T}{L}}
				..(((((((..((....)).)))))))...((((((....((((((...
				((...((((((....))))))..))...))))))(((...(.((((((.
				.{\icbox{\phantom{{AUUGCACUCC}}}}..)))))).)..))).
				..[.[[[[[...))))))((((...(((....)))..))))......]]
				]]]]..((.(((((....))))).....))..
				{\textcolor{gray}{..........}}} 
		\end{tabularx}
		\caption{Sequence and secondary structure taken from GISSD \parencite{zhou_gissd_2008}.
		}\label{fig:azodata:b}
	\end{subfigure}
	\caption[Sequence and Structure Data]{
		Sequence and structure data used for the design of \textit{Azoarcus}-like sequences. While
		\begin{enumerate*}[label={(\alph*)}, font={\bfseries}]
			\item was taken from \parencite{mustoe_secondary_2016}, the data corresponds to the structure of \texttt{PDB 1U6B} obtained through crystallography \parencite{adams_crystal_2004}.
			\item was taken from GISSD and lacks a region added to \ref{fig:azodata:a} for the crystallography (\icbox{boxed}) \parencite{adams_crystal_2004}.
		\end{enumerate*}
		Positions in gray indicate truncations dictated by requirements for experimental assays. Lower-case letters indicate positions not part of the intron.
	}\label{fig:azodata}
\end{figure}


With experimental assays in mind at later stages, data from the Group I Intron Sequence and Structure Database (GISSD) was taken into account \parencite{zhou_gissd_2008}.
The sequence data from GISSD differed from the sequence data used in the crystallography mentioned above, primarily in the omission of the binding site for the co-crystallized protein (\autoref{fig:azodata:b}).


Moreover, positions of the exons adjacent to the intron were disregarded at later stages in this work (\autoref{fig:azodata:b}).

\end{document}
